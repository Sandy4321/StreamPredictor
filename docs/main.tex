\documentclass[fleqn,10pt]{olplainarticle}
% Use option lineno for line numbers 

\usepackage{algorithm}
\usepackage[noend]{algpseudocode}

\title{Stream Predictors}

\author[1]{Abhishek Rao}
%\author[2]{Second Author}
%\affil[1]{Unaffiliated}
%\affil[2]{Address of second author}

\keywords{Sequence modelling, language modelling, hierarchical hidden markov model}

\begin{abstract}
We try to model sequential structured data like natural text using a combinatino of Hierarchical Patterns and Evolution. 
The aim is to get a probabilistic, language independent model for sequential data. 
During training we learn which pattern comes after each pattern similar to a bigram. 
Then each transition from pattern to pattern is made as a new pattern.
Many spurious patterns are formed but only ones that are repeated are retained. Performance is calculated by measuring the perplexity for words from some Enlgish novels.
\end{abstract}

\begin{document}

\flushbottom
\maketitle
\thispagestyle{empty}

\section*{Intoduction}
\subsection{Hierarchical Pattern Formation}
The main concept used is somewhat similar to Hierarchical Hidden Markov Model  (\textbf{HHMM}). The idea is that every transition from pattern A to pattern B can itself be modelled as pattern AB. But the inference method is different.

\begin{figure}[ht]
\centering
\includegraphics[width=0.7\linewidth]{pop}
\caption{Pops. Pattern of Patterns. The bottom row shows the given sequence along time for modelling. In this case it is the characters
cathat. The second row shows the patterns learned in the first pass. The third row patterns learned in the second pass and so on. Note: not all patterns (e.g. "ha") are shown in the diagram.}
\label{fig:view}
\end{figure}

Suppose that our collection of pattern consists of \(\{A,B,C,...Z\}\). i.e. All the Alphabets. Suppose in the first iteration we are given the sequence "CATHAT". 

1. In the first pass we would form patterns like this 
CA, AT, TH, HA, AT. Now our pattern collection would consists of  \(\{A,B,C,...Z, CA, AT, TH, HA, AT\}\).

2. In the next pass if we encounter the same sequence "CATHAT" we would split the given sequence in terms of largest existing patterns to form the transformed sequence of "CATHAT" = \(\{"CA", "TH", "AT"\}\). 
And we would form new patterns by combining 2 at a time the transformed sequence. \( \{ "CATH", "THAT"\}\). And so on. 

\subsection{ Evolution of Patterns}
With the above formation lots of spurious patterns will be formed. We need to keep only useful patterns and discard unused ones. So Every time we encounter a pattern we "feed" it and it's strength increases. Strength is just an integer number associated with each pattern. As time increases we decrement strength of all patterns. If strength falls below certain threshold we cull the patterns. Hence patterns that are not repeated will die out. Also when we feed a pattern it's children are also given a share. It is essential that all formed pattern are given sufficient time before being killed as they maybe useful later on as children of other more complicated patterns.

\subsection{Categories}
We can find words that occur in similar position and categorize them into classes. 
e.g. If we see the sentences \textit{He lives in a house}and \textit{She lives in a house} we can conclude that He and She are similar. 
This way we can find classes of words. This is similar to \cite{brown1992class}

\section*{Algorithm}
The algorithm for Stream Predictors is given in Table Algorithm \ref{sp_algo}.
\begin{algorithm}
\caption{Stream Predictors Algorithm}
\label{sp_algo}
\begin{algorithmic}[1]
\State Given a sequence of symbols $\textbf{X} = x_0, x_1, \ldots x_N$

\Procedure {Train}{Build sequence model from given sequence}
\State Assume currently we have a collection of Patterns denoted by $P$ and $p_i \in P$ are the individual patterns in $P$.
\State We initialize $p_i$ by the unique symbols in \textbf{X}. i.e. $P = Set(\textbf{X})$
\State $index = 1, p_{current} = x_0$
\While {$index<N$}
\State $p_{next} = GetNextPattern(x_{index}.. x_N)$
\State $SetPattern(p_{current}, p_{next})$
\State $p_{current} = p_{next}$, $index = index + length(p_{next})$
\EndWhile
\EndProcedure
\Procedure {$GetNextPattern(x_0,x_1, \ldots x_N)$}{Gets the next pattern in stream present in P}
\State Find the longest pattern $p_i \in P$ such that \(p_i = x_0, x_1, \ldots x_l\)
\State Return $p_i$
\EndProcedure
\Procedure {$SetPattern(p_{current}, p_{next})$}{Creates or strengthens pattern by combining two patterns}
\State Find the longest pattern $p_i \in P$ such that $p_i = x_{0}, x_1, \ldots x_l $
\State Return $p_i$
\EndProcedure
\For{$m=1,2 \dots L-1$ }
\State Calculate $x_L = <\textbf{x}, f_{1}(\textbf{x}), f_{2}(\textbf{x}), \dots f_{L-1}(\textbf{x})>$, where $x_L \in \mathbb{R}^{D+L-1}$
\EndFor
\State Train $f_L(\textbf{x}_L) = h(\sum_1^{D+L-1}w_i.x_i + b_L)$ using the SVM algorithm, parameter $C$ and L1 regularization. Here $h()$ is the activation function like sigmoid or tanh.
\Procedure{Predict}{}
\State $y_i^L = sgn(f_L(\textbf{x}_i))$
\EndProcedure
\end{algorithmic}
\end{algorithm}


\section*{Code}
The given code uses the above concepts for text. 

\subsection{Pop}
Each unit of pattern is called Pop (Pattern of Pattern). It consists of an unrolled pattern which is the complete text representation of the pattern. It uniquely identifies each pop. Each pop also contains two children called first component and second component. 
The following relationship holds 
*Pop.unrolled pattern = Pop.first component.unrolled pattern + Pop.second component.unrolled pattern*
Where the + symbol denotes concatenation. 

\subsection{Pop Manager}
Consists of collection of Pops as a dictionary where the keys are the string associated with the \textit{Pop.unrolled pattern} and the values are the corresponding Pops.

Some of the methods of this class are 

1. *Train()* Takes a string as input and trains the model. Essentialy forms a collection of patterns.
2. *Generate()* Using the patterns collection generates a stream of likely text. 
Thanks for using Overleaf to write your article. Your introduction goes here! Some examples of commonly used commands and features are listed below, to help you get started.

\section*{Experiment}
For the experiment, given a collection of text like novel, we split the words into test and training set. We train the model on the training set and then calculate the perplexity in the test set.













\bibliography{sample}

\end{document}

